%% XeLaTeX

\documentclass[a4paper,10pt]{article}

\usepackage[cp1250]{inputenc}
\usepackage[polish]{babel}
\usepackage{fontspec} % For loading fonts

\usepackage{geometry}
\geometry{a4paper}

%%% SECTION TITLE APPEARANCE
\usepackage{sectsty}
\allsectionsfont{\sffamily\mdseries\upshape} % (See the fntguide.pdf for font help)


%ustawienia rozmiarów: tekstu, stopki, marginesów 
\setlength{\textwidth}{\paperwidth}
\addtolength{\textwidth}{-3cm}
\setlength{\textheight}{\paperheight}
\addtolength{\textheight}{-5cm}
\setlength{\oddsidemargin}{-1.04cm} % domyślnie jest 1 cal = 2.54 cm, stąd -0.04 da margines 2.5cm
\setlength{\evensidemargin}{-1.04cm} % domyślnie jest 1 cal = 2.54 cm, stąd -0.04 da margines 2.5cm
\topmargin -3.25cm 
\footskip 1.4cm 


%%%%%%%%%%%%
\title{Opis zadania projektowego}
\author{Michał Czapowski 181225\\Grzegorz Grzegorczyk 181121\\grupa ?}


\begin{document}
\maketitle


\section{Temat i cel projektu}

\section{Opis działania systemu}
\section{Założenia przyjęte podczas realizacji systemu}
\section{Wykorzystane technologie, narzędzia projektowania oraz implementacji systemu}
\begin{enumerate}
\item{Java}
\item{Hibernate}
\item{Oracle}
\item{Vaadin}
\item{MVC}
\item{Tomcat/JBoss/GlassFish}
\end{enumerate}

\section{Schemat komunikacji, struktura systemu}

\section{Literatura}

\end{document}
